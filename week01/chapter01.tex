\chapter{리눅스 소개}

\section{모던 환경이란 무엇인가?}
작성 시점인 2024년 기준 리눅스는 다양한 환경에서 사용된다.

\begin{itemize}
    \item 모바일 디바이스 \newline 
        안드로이드 OS는 리눅스의 변형이다.
    \item 클라우드 컴퓨팅 \newline  
        리눅스는 CSP에서 제공하는 서버의 OS로 많이 사용된다.
    \item IoT \newline
        리눅스는 사물 인터넷 OS로도 사용된다.
    \item 프로세스 아키텍처의 다양성 \newline 
        전통의 x86-64 이외에도 ARM, RSIC-V 같은 ISA도 지원된다.
    \item 컨테이너 인프라 \newline
        Kubernetes, Docker로 대표되는 컨테이너 인프라에서 기본적으로 사용된다.
\end{itemize}
\newpage


\section{리눅스 역사}
\subsection*{90년대}
1991년 리누스 토르발스에 의해 시작되었으며 수많은 기여자의 도움으로 리눅스 1.0.0은 3년이 되기도 전에 릴리즈되었다.
\newline
이 시점에서 Unix/GUN 소프트웨어를 실행할 수 있는 운영체제라는 최초 목표는 달성되었다.
또한, 이 기간에 최초의 상용 리눅스인 Red Hat 리눅스가 등장했다.

\subsection*{2000 - 2010}
여러 기업에서 리눅스를 본격적으로 도입하며 폭팔적으로 성장하는 시기이다.\newline
이 시기부터 Linux는 Unix, Windows Server 같은 경쟁 서버보다 점유율 우위를 점하기 시작한다.
또한, 여러 배포판(distro)가 출시되며 경쟁했다.

\subsection*{2010 이후}
단순 서버용 이상으로 확장되어 IoT, 모바일 디바이스에서도 사용한다.\newline
또한, 컨테이너 인프라가 본격적으로 사용되기 시작했다.\newline
그리고, 상용 서버 배포판은 사실상 Red Hat, Debian 기반으로 수렴했다.

\begin{flushleft}
    보다 상세한 한글 자료는 \href{https://zdnet.co.kr/view/?no=20200826135027}{리눅스 29돌 사건으로 보는 그 역사}를 참고하라.    
\end{flushleft}



\section{리눅스 배포판}
\textbf{리눅스 커널}은 \textbf{단순 시스템 콜과 디바이스 드라이버의 집합}을 의미한다.\newline
\textbf{리눅스 배포판}은 \textbf{커널과 관련 구성요소의 전체 번들}을 의미한다. 
관련 구성요소에는 패키지 관리자, 파일시스템 레이아웃, init system, shell이 포함된다.
\newpage


\section{리소스 가시성}
리눅스는 리소스의 \textbf{전역 보기}(global view)를 지원한다.\newline
여기서 \textbf{전역보기}는 무엇이고 \textbf{전역보기의 반대}는 무엇이며 \textbf{리소스}는 무엇인가?

\subsection*{리소스란?}
\textbf{리소스}는 \textbf{소프트웨어 실행을 지원하는 데 사용할 수 있는 모든 것}으로 간주될 수 있다.\newline
예를 들면 다음과 같은 것이 리소스로 간주될 수 있다.

\begin{itemize}
    \item 하드웨어
    \item 파일시스템
    \item HDD/SSD
    \item process
    \item device
    \item routing table(Network)
    \item 자격증명(credential)
\end{itemize}

\subsection*{전역 리소스}
예를 들어 HDD 용량 확인, 특정 파일 읽기, ps -ef로 pid 확인하기 같은 것은 여러 계정에서 동일하게 볼 수 있다.
\newline
리눅스에서 할 수 있는 대부분의 것이 전역에서 되므로 리눅스의 모든 것은 전역일 것이라 기대할 수 있지만 
실제로 그렇지 않다.


\subsection*{로컬 리소스}
모든 리소스가 전역인 것은 아니다. 예를 들면 namespace를 통해 리소스의 로컬보기를 지원할 수 있다.
또한, 특정 프로세스가 메모리 리소스를 고갈시키는 것을 방지하기 위해 
메모리 소비를 제한하는 것도 메모리가 전역 리소스로 사용하지 않을 수 있다든 것을 보여준다.\newline
이처럼 리소스의 사용을 제한하는 것은 프로세스의 격리의 일종이며 
리눅스에서는 \textbf{cgoup}이라는 커널 기능을 사용해 이러한 종류의 격리를 제공한다.\newline
